%%%%%%%%%%%%%%%%%%%%%%%%%%%%%%%%%%%%%%%%%%%%%%%%%%%%%%%%%%%%%%%%%%%%%%
% LaTeX Example: Project Report
%
% Source: http://www.howtotex.com
%
% Feel free to distribute this example, but please keep the referral
% to howtotex.com
% Date: March 2011 
% 
%%%%%%%%%%%%%%%%%%%%%%%%%%%%%%%%%%%%%%%%%%%%%%%%%%%%%%%%%%%%%%%%%%%%%%
% How to use writeLaTeX: 
%
% You edit the source code here on the left, and the preview on the
% right shows you the result within a few seconds.
%
% Bookmark this page and share the URL with your co-authors. They can
% edit at the same time!
%
% You can upload figures, bibliographies, custom classes and
% styles using the files menu.
%
% If you're new to LaTeX, the wikibook is a great place to start:
% http://en.wikibooks.org/wiki/LaTeX
%
%%%%%%%%%%%%%%%%%%%%%%%%%%%%%%%%%%%%%%%%%%%%%%%%%%%%%%%%%%%%%%%%%%%%%%
% Edit the title below to update the display in My Documents
%\title{Project Report}
%
%%% Preamble
\documentclass[paper=a4, fontsize=11pt]{scrartcl}
\usepackage[T1]{fontenc}
\usepackage{fourier}

\usepackage[polish]{babel}															% English language/hyphenation
\usepackage[protrusion=true,expansion=true]{microtype}	
\usepackage{amsmath,amsfonts,amsthm} % Math packages
\usepackage[pdftex]{graphicx}	
\usepackage{url}
\usepackage{listings}

\graphicspath{{img/}}             % Katalog z obrazkami.


%%% Custom sectioning
\usepackage{sectsty}
\allsectionsfont{\centering \normalfont\scshape}


%%% Custom headers/footers (fancyhdr package)
\usepackage{fancyhdr}
\pagestyle{fancyplain}
\fancyhead{}											% No page header
\fancyfoot[L]{}											% Empty 
\fancyfoot[C]{}											% Empty
\fancyfoot[R]{\thepage}									% Pagenumbering
\renewcommand{\headrulewidth}{0pt}			% Remove header underlines
\renewcommand{\footrulewidth}{0pt}				% Remove footer underlines
\setlength{\headheight}{13.6pt}


%%% Equation and float numbering
\numberwithin{equation}{section}		% Equationnumbering: section.eq#
\numberwithin{figure}{section}			% Figurenumbering: section.fig#
\numberwithin{table}{section}				% Tablenumbering: section.tab#


%%% Maketitle metadata
\newcommand{\horrule}[1]{\rule{\linewidth}{#1}} 	% Horizontal rule

\title{
		%\vspace{-1in} 	
		\usefont{OT1}{bch}{b}{n}
		\normalfont \normalsize \textsc{Politechnika Warszawska Wydział EiTI Instytut Informatyki} \\ [25pt]
		\horrule{0.5pt} \\[0.4cm]
		\huge Dokumentacja ALHE SNDLib sieć Polska \\
		\horrule{2pt} \\[0.5cm]
}
\author{
		\normalfont 								\normalsize
        Kacper Kula \& Wojciech Sitek \\[-3pt]		\normalsize
        \today
}
\date{}


%%% Begin document
\begin{document}
\maketitle
\section{Zadanie projektowe}

Przy użyciu Algorytmu Ewolucyjnego zaprojektować sieć teleinformatyczną minimalizującą liczbę użytych systemów teletransmisyjnych o różnej modularności m (m = 1, m > 1 i m >> 1). Sieć opisana za pomocą grafu G = (N, E), gdzie N jest zbiorem węzłów, a E jest zbiorem krawędzi. Funkcja pojemności krawędzi opisana jest za pomocą wzoru, określonego w zadaniu. Zbiór zapotrzebowań D, pomiędzy każdą parą węzłów opisuje macierz zapotrzebowań i jest dany. Dla
każdego zapotrzebowania istnieją co najmniej 3 predefiniowane ścieżki. Sprawdzić,
jak wpływa na koszt rozwiązania agregacja zapotrzebowań, tzn. czy zapotrzebowanie realizowane jest na jednej ścieżce (pełna agregacja), czy dowolnie, na
wszystkich ścieżkach w ramach zapotrzebowania (pełna dezagregacja). Dobrać
optymalne prawdopodobieństwo operatorów genetycznych oraz liczność populacji. Dane pobrać ze strony http://sndlib.zib.de/home.action, dla sieci polska.

\section{Wyjaśnienie pojęć}
\label{section:explanation}

\subsection{Modularność}
Modularność jest to ... 

\subsection{Algorytm Ewolucyjny}

Zgodnie z wykładami prof. Jarosława Arabasa, Algorytm Ewolucyjny opisuje się za pomocą algorytmu \ref{algorithm:evolutionary}. Na algorytm składają się następujące funkcjonalności:
\begin{enumerate}
    \item inicjalizacja populacji,
    \item główna pętla algorytmu ewolucyjnego,
    \item mutacja (ang. \textit{mutation}),
    \item krzyżowanie (ang. \textit{crossover}),
    \item selekcja (ang. \textit{selection}, oznaczane jako \textit{select}),
    \item warunek zatrzymania,
    \item funkcja celu.
\end{enumerate}

\section{Założenia projektu}

Projekt wykonywany jest w ramach przedmiotu Algorytmy Heurystyczne (ALHE) w semestrze zimowym 2020 na Wydziale EiTI Politechniki Warszawskiej. Prowadzącym projekt jest dr inż. Stanisław Kozdrowski.

Implementacja projektu jest wykonywana w języku Python. Algorytm ewolucyjny posiada własną implementację i nie jest zaczerpnięty z bibliotek zewnętrznych języka Python. Biblioteki narzędziowe języka Python, którymi się wspomagano, to między innymi biblioteki:
\begin{itemize}
    \item xml - do przeprowadzenia parsowania pliku XML do obiektów Python (rozdział \ref{section:preparing})
    \item tqdm - ukazywania paska postępu
    \item random - do generacji liczb pseudolosowych (użyto ziarna (ang. \textit{seed}))
    \item logging - do logowania informacji pomocniczych w czasie działania programu
    \item json - do zapisu i odczytu plików JSON.
\end{itemize}

\section{Przygotowanie danych}
\label{section:preparing}

Dla sieci ,,polska'', pobrano plik XML oraz TXT ze wszystkimi informacjami dla danych dotyczących terytorium Polski. Następnie, przeprowadzono analizę budowy plików oraz znaczenia poszczególnych terminów. Wyeliminowano z programu dane nieistotne dla rozwiązywanego problemu. Analiza znaczenia terminów przeprowadzona jest w rozdziale \ref{section:explanation}.

Następnie, zbudowano parser, konwertujący wybrane części pliku XML do obiektów języka Python. Budowa słowników i list, w których były przechowywane informacje o sieci, została przedstawiona na poniższym przykładzie:
\lstset{language=Python}
\begin{lstlisting}
nodes = ['Gdansk', 'Bydgoszcz', ...]

link_keys = [
    (0, 1),
    (0, 2),
    (1, 3),
]

links_array = [
    Link_0_1_data,
    Link_0_2_data,
    ...,
    Link_1_3_data,
    ...,
]

Link_a_b_data = {
    'setupCost': 156.0,
    'capacity0': 155.0,
    'capacity1': 622.0,
    'cost0': 156.0,
    'cost1': 468.0,
}

demand_array = [
    Demand_0_1_data,
    Demand_0_2_data,
    ...
]

Demand_a_b_data = {
    'demand': 195.00,
    'admissiblePaths': [
        [(0,2), (1,2)],
        [(0,10), (1,10)],
        [(0,2), (2,9), (7,9), (1,7)],
        ...
    ]
}

demand_keys = [
    (0, 1),
    (0, 2),
    (1, 2),
]

\end{lstlisting}

\section{Charakterystyka implementacji algorytmu}

\subsection{Mutacja}

itp

\section{Analiza wyników}

% tabela

%%% End document
\end{document}